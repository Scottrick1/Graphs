% chap4.tex (Week 4 questions)

% Note that the text in the [] brackets is the one that will
% appear in the table of contents, whilst the text in the {}
% brackets will appear in the main thesis.

\chapter[Week 4]{Week 4}

\section{Q 4.1.1}
For this question we use corollary 4.1. Being able to draw the figures without lifting the pen from the page is equivalent to the figure being Eulerian. Clearly the first and third figures have only 2 vertices of odd degree and so are Eulerian. The second figure on the other hand has 4 vertices of odd degree and so is non-Eulerian.

\section{Q 4.1.2}
BLOCK FISH!

\section{Q 4.1.4}
Seems intuitive, but how to show?\\
2.2.6 tells us that if G has no edges with odd degree then G has no cut edges. If G has no cut edges, then there are at least 2 disjoint paths between any two vertices; ie G is 2 edge-connected. It follows then that we can consider a pair of disjoint paths to be a cycle. If we only consider the cycles that are \emph{simplest cycles} and th\\
\\
\\
G has no vertices of odd degree means that all the vertices of G are of even degree, and hence, has at least one cycle say $C_1$. The removal of the edges in $C_1$ results in a spanning subgraph of $G$ denoted as $G_1$. If $G_1$ has no edges then we have the edges of $C_1$ is as that of $G$ and hence, the edges of $G$ can be partitioned into cycles. Otherwise, $G_1$ contains a cycle say $C_2$, since all the vertices of $C_1$ are of even degree. Again, removing the edges of $C_2$ results in a spanning subgraph of $G_2$, in which every vertex still has an even degreee, continue this process of removing the edges of the cycles, and after, say $m$ steps we obtain finally a totally disconnected graph, sy $G_m$ and we have the edge disjoint cycles, $C_1,C_2,\dots,C_m$ whose union will be graph G.\\

\section{Q 4.2.2}
Bipartite 13 and 14, color the blocks adjacent. ie no you can't - poor mouse


\section{Q 4.2.9}
I don't know.

\section{Q 4.4.1}


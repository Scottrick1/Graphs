% chap6.tex (Week 6 questions)

% Note that the text in the [] brackets is the one that will
% appear in the table of contents, whilst the text in the {}
% brackets will appear in the main thesis.

\chapter[Week 6 - Edge Colorings]{Week 6 - Edge Colorings}

\section{Q 6.1.1}
	$K_{m,n}$ is a complete bipartite graph with partitions of size $m$ and $n$. We know from Theorem 6.1 that a bipartite graph has $\chi^{\prime} = \Delta$. So we know that an edge coloring exists. An appropriate edge coloring can be constructed thusly\\
	SHIFTING


\section{Q 6.1.2}
	Show that $\chi^{\prime} \neq 3$\\

	


\section{Q 6.1.5}
	I have no idea what Bensons proof is...

\begin{figure}[H]
	\centering
	\begin{tikzpicture}[auto,node distance=3cm,
	thick,main node/.style={circle,draw,font=\sffamily\Large\bfseries}]
  	
    \node[main node] (1) {$v_1$};
    \node[main node] (2) [left = 3cm of 1]  {};
    \node[main node] (3) [right = 3cm of 1] {};
    \node[main node] (4) [below = 3cm of 1] {$v_2$};
    \node[main node] (5) [left = 3cm of 4] {}; {};
    \node[main node] (6) [right = 3cm of 4] {};] {};

    \path[draw,thick,draw=orange]
    (1) edge node {} (2)
    ;
    \path[draw,thick,draw=purple]
    (1) edge node {} (3)
    ;
    \path[draw,thick,draw=green]
    (4) edge node {} (5)
    ;
    \path[draw,thick,draw=blue]
    (4) edge node {} (6)
    ;
    \path[draw,thick,draw=black,dashed]
    (1) edge node {} (4)
    ;
	\end{tikzpicture}
\end{figure}


\section{Q 6.2.1}
	The $2n$ case:\\ Consider a $2n-1$ - gon with $2n^{th}$ vertex in the middle. 

	\begin{figure}[H]
	\centering
	\begin{tikzpicture}[auto,
	thick,main node/.style={circle,draw,font=\sffamily\Large\bfseries}]
  	
    \node[main node] (1) [label = $2n-2$]{};
    \node[main node] (2) [below right = 0.5cm and 2cm of 1,label = $2n-1$] {};
    \node[main node] (3) [below right = 1.5cm and 1.5cm of 2, label = $1$] {};
    \node[main node] (4) [below right = 2cm and 0.5cm of 3, label = $2$] {};
    \node[main node] (5) [below left = 9cm and 5cm of 1, label = below left:$n$]  {};
    \node[main node] (6) [below right = 1cm and 2cm of 5, label = below:$n-1$] {};
    \node[main node] (7) [above left = 2cm and 1cm of 5, label = left:$n+1$] {};
    \node[main node] (8) [below left = 5 cm and 1cm of 1, label = left:$2n$] {};

    \path[draw,thick, dashed]
    (1) edge node {} (2)
    (2) edge node {} (3)
    (3) edge node {} (4)
    (4) edge[bend left=55] node  {} (6)
    (5) edge node {} (6)
    (5) edge node {} (7)
    (7) edge[bend left=55] node  {} (1)
    (5) edge node {} (8)
    ;

	\end{tikzpicture}
\end{figure}

Matchings!\\

\begin{figure}[H]
	\centering
	\begin{tikzpicture}[auto,
	thick,main node/.style={circle,draw,font=\sffamily\Large\bfseries}]
  	
    \node[main node] (1) [label = $2n-2$]{};
    \node[main node] (2) [below right = 0.5cm and 2cm of 1,label = $2n-1$] {};
    \node[main node] (3) [below right = 1.5cm and 1.5cm of 2, label = $1$] {};
    \node[main node] (4) [below right = 2cm and 0.5cm of 3, label = $2$] {};
    \node[main node] (5) [below left = 9cm and 5cm of 1, label = below left:$n$]  {};
    \node[main node] (6) [below right = 1cm and 2cm of 5, label = below:$n-1$] {};
    \node[main node] (7) [above left = 2cm and 1cm of 5, label = left:$n+1$] {};
    \node[main node] (8) [below left = 5 cm and 1cm of 1, label = left:$2n$] {};

    \path[draw,thick, dashed]
    (1) edge node {} (2)
    (2) edge node {} (3)
    (3) edge node {} (4)
    (4) edge[bend left=55] node  {} (6)
    (5) edge node {} (6)
    (5) edge node {} (7)
    (7) edge[bend left=55] node  {} (1)
    (5) edge node {} (8)
    ;
        \path[draw,ultra thick, red]
    (1) edge node {} (4)
    (2) edge node {} (3)
    (6) edge node {} (7)
    (5) edge node {} (8)
    ;

	\end{tikzpicture}
\end{figure}

	The $2n-1$ case:\\ Now remove the middle vertex. We no longer have $2n-1$ perfect matchings, but we can still consider the problem in a similar fashion. We still consider the parallel matchings, except now we have an odd vertex in the polygon that is unsaturated. As we rotate the matchings around $2n-1$ times, I don't know what I"m saying waaaaaah.

	\begin{figure}[H]
	\centering
	\begin{tikzpicture}[auto,
	thick,main node/.style={circle,draw,font=\sffamily\Large\bfseries}]
  	
    \node[main node] (1) [label = $2n-2$]{};
    \node[main node] (2) [below right = 0.5cm and 2cm of 1,label = $2n-1$] {};
    \node[main node] (3) [below right = 1.5cm and 1.5cm of 2, label = $1$] {};
    \node[main node] (4) [below right = 2cm and 0.5cm of 3, label = $2$] {};
    \node[main node] (5) [below left = 9cm and 5cm of 1, label = below left:$n$]  {};
    \node[main node] (6) [below right = 1cm and 2cm of 5, label = below:$n-1$] {};
    \node[main node] (7) [above left = 2cm and 1cm of 5, label = left:$n+1$] {};


    \path[draw,thick, dashed]
    (1) edge node {} (2)
    (2) edge node {} (3)
    (3) edge node {} (4)
    (4) edge[bend left=55] node  {} (6)
    (5) edge node {} (6)
    (5) edge node {} (7)
    (7) edge[bend left=55] node  {} (1)
    ;
        \path[draw,ultra thick, red]
    (1) edge node {} (4)
    (2) edge node {} (3)
    (6) edge node {} (7)
    ;

	\end{tikzpicture}
\end{figure}

\section{Q 6.3.1}
	\begin{enumerate}[(a)]
	\item The number of periods needed in the week can be simply taken as $max\{\sum_i (p_{ij} , \sum_j (p_{ij}\}$. We then assume that we want an even distribution period across days and so divide this value by the number of days in the week, $5$.
	\item $\frac {\sum_{i,j} p_{ij}} {5 \times 8}$
	\end{enumerate}

\section{BONUS 6.2.6}
	\begin{enumerate}[(a)]
	\item Using Vizing's theorem (6.2) show that $\chi^\prime(G \times K_2) = \Delta(G \times K_2)$.\\
	It is immediately apparent that $\Delta(G \times K_2) = \Delta(G) + \Delta(K_2)$, in fact, this is clear for any two loopless graphs $\Delta(G \times H) = \Delta(G) + \Delta(H)$. We won't necessarily use that here, but it's good for intuition.\\
	$K_2$ is a simple stick - 2 vertices connected by an edge.\\
	Consider $G$ to be a graph with $V = {v_1,v_2,v_3}, E = {v_1v_2,v_2v_3}$, as below.\\

\begin{figure}[H]
	\centering
	\begin{tikzpicture}[auto,
	thick,main node/.style={circle,draw,font=\sffamily\Large\bfseries}]
  	
    \node[main node] (1) {$x_1$};
    \node[main node] (2) [below right = 2cm and 1.5cm of 1]  {$x_2$};
    \node[main node] (3) [below = 4cm of 1] {$x_3$};

    \path[draw,ultra thick,blue!40]
    (1) edge node {} (2);
    \path[draw,ultra thick,orange]
    (2) edge node {} (3);
	\end{tikzpicture}
\end{figure}

And just there are no doubts as to what we're working with, here's $K_2$:

\begin{figure}[H]
	\centering
	\begin{tikzpicture}[auto,
	thick,main node/.style={circle,draw,font=\sffamily\Large\bfseries}]
  	
    \node[main node] (1) {$y_1$};
    \node[main node] (2) [right = 2cm of 1]  {$y_2$};

    \path[draw,ultra thick,green]
    (1) edge node {} (2);
	\end{tikzpicture}
\end{figure}

	Now let's look at the product $G \times K_2$\\

\begin{figure}[H]
	\centering
	\begin{tikzpicture}[auto,
	thick,main node/.style={circle,draw,font=\sffamily\Large\bfseries}]
  	
    \node[main node] (1) {$x_1,y_1$};
    \node[main node] (2) [below right = 2cm and 1.5cm of 1]  {$x_2,y_1$};
    \node[main node] (3) [below = 4cm of 1] {$x_3,y_1$};

    \node[main node] (4) [right = 5cm of 1] {$x_1,y_2$};
    \node[main node] (5) [below right = 2cm and 1.5cm of 4] {$x_2,y_2$};
    \node[main node] (6) [below = 4cm of 4] {$x_3,y_2$};

    \path[draw,ultra thick,blue!40]
    (1) edge node {} (2)
    (4) edge node {} (5);
    
	\path[draw,ultra thick,orange]
    (2) edge node {} (3)
    (5) edge node {} (6);

    \path[draw,ultra thick,green]
    (1) edge node {} (4)
    (2) edge node {} (5)
    (3) edge node {} (6)
    ;

	\end{tikzpicture}
\end{figure}


	From (6.2) we know that for any loopless graph $G$ that we could choose, $\chi^\prime(G \times K_2) = \Delta(G \times K_2)$ or $\chi^\prime(G \times K_2) = \Delta(G \times K_2) + 1$. All we need to do is show that the $+ 1$ is unnecessary.\\

	\item Deduce that if $H$ is nontrivial with $\chi^\prime(H) = \Delta(H)$, then $\chi^\prime(G \times H) = \Delta(G \times H)$.\\

	\end{enumerate}


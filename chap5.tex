% chap5.tex (Week 5 questions)

% Note that the text in the [] brackets is the one that will
% appear in the table of contents, whilst the text in the {}
% brackets will appear in the main thesis.

\chapter[Week 5 - Matchings]{Week 5 - Matchings}


\section{Q 5.1.1}
\begin{enumerate*}[(a)]

\item A 2-cube is a square, we can think of a 3-cube as being an extrusion of the square into 3 dimensional space. Similarly, a 4-cube is an extrusion of a 3-cube into a 4 dimensional space, etc, etc. A square has a perfect matching as we can take any two opposite edges of the square as matchings and all 4 vertices become saturated. An extrusion can be thought of as shifting a copy of the template along the new dimensions and connecting each vertex with an edge. Since both the face that was the template and the face that is the shifted component are both squares, they both have a perfect matching, and there are no other edges and so the cube has a perfect matching. The exact same idea can be used for every step k to k+1, and so every k-cube has a perfect matching.
\item $K_{2n}$ is complete so it has $2n$ vertices and every vertex has $2n-1$ incident edges, i.e.  edges.

$K_{n,n}$ is complete bipartite so it has 2 partitions each with $n$ vertices and every vertex has $n$ incident edges, i.e. $2n$ vertices and  edges.

\section{Q 5.1.2}

A tree has at least least two vertices with $d(v_i) = 1, \, (i = 0,1) $, and so a perfect matching must include the edges incident to these vertices. Since a tree is acyclic, the matchings must be every alternating edge on the path between \emph{any} $v_x \text{and} v_y, x,y = 1,\ldots,n$, thus saturating every vertex along the path. If this fixed end alternating pattern is not possible (i.e. there are an even number of edges the compose the path between any two degree 1 vertices), then a perfect matching does not exist. There is no possible other perfect matching distinct from the one found by this method.


\section{Q 5.1.5}


\section{Q 5.3.4}
% Questions from week 2

\chapter[Week 2 - Trees and Cuts]{Week 2 - Trees and Cuts}

\section{Q 2.1.5}
\begin{enumerate}[(a)]
\item Since the number of edges of $G$ is constrained to $v-1$ we can construct any arbitrary connected graph by sequentially adding a vertex and edge pair to a kernel vertex. They must be added as a pair otherwise the \# edges constraint would be violated. Creating a cycle requires adding an edge between two vertices that already exist. Since we cannot at any stage do this, $G$ must be acyclic.\\
Since $G$ is connected and acyclic, it is a tree.

\item $G$ is acyclic and so it is a forest. Because it is a forest, we know that $\varepsilon(G) = v(G) - \omega(G)$ and we also have the condition that $\varepsilon(G) = v(G) - 1$. It is clear then that $\omega(G) = 1$ and therefore we have only 1 component which means that $G$ is connected.\\
Since $G$ is acyclic and connected, it is a tree.

\item $G$ is a tree which means it is acyclic and connected by definition.
\end{enumerate}


\section{Q 2.2.4}
\begin{enumerate}[(a)]
\item Since the textbook does not define a Maximal Forest, let's define it as follows:\\
$F$ is a maximal forest of $G$ when for every component $H$ of $G$, $F \cap H$ is a spanning tree of $H$.\\
Thus the solution is: by the definition above.

\item Each component of the forest is a tree with $v_{sub}-1$ edges where $\sum_{\omega}(v_{sub}) = v(G)$, so the total number of edges of the forest epsilon(F) is simply $\sum_{\omega}(v_{sub}) - \omega* 1$, i.e. $\varepsilon(F) = v(G) - \omega(G)$.
\end{enumerate}

\section{Q 2.2.6}
\begin{enumerate}[(a)]
\item Assuming that there IS a cut edge, imagine two disjoint subgraphs A and B of G s.t. there is only one edge e, not in A or B, connecting the two subgraphs, i.e. $A+B+e = G$. Now consider the vertex of $A$ that is the vertex in $G$ incident with $e$, since $e$ is not included in $A$, this vertex must have an odd degree $(even number - 1 = odd number)$. However, since every other vertex in $A$ must have an even degree, this is clearly (explain!) impossible so we have a contradiction.

\item Yuck. To be done.

\end{enumerate}

\section{Q 2.3.1}
\begin{enumerate}[(a)]
\item Clearly the cut edge must incident to two vertices, if we were to remove a vertex with $\delta>1$ (at least one of the two vertices must satisfy this, if not both) it would result in a loss of the edge as well. At a minimum such a cut vertex must increase the number of components by 1, but if it is also incident with other cut edges (which would also disappear) then correspondingly more components are created.

\item Consider a graph with a central node that is connected to all other vertices, and each pair of outer vertices is also connected. Such a graph has no cut edge but the central node is a cut vertex - removing it would increase the number of components drastically.

\end{enumerate}

